% Currently this document is written in German
% !TeX spellcheck = de_DE

%Ensure that all odl school LaTeX habits are remarked
\RequirePackage[l2tabu, orthodox]{nag}
%Neue deutsche Trennmuster
%Siehe http://www.ctan.org/pkg/dehyph-exptl und http://projekte.dante.de/Trennmuster/WebHome
%Nur für pdflatex, nicht für lualatex
\RequirePackage[ngerman=ngerman-x-latest]{hyphsubst}
\documentclass{lni}
% in Englisch stattdessen:
%\documentclass[english]{lni}







%% Some packages, no need to be adabted

% enable copy and paste of ligatures (e.g., in "workflow" and umlauts)
\usepackage{cmap}

%Überschrift des Literaturverzeichnisses
%Only in German
\iflnienglish
\else
\renewcommand{\refname}{Literaturverzeichnis}
\fi

%Enable input of umlauts using UTF-8.
\usepackage[utf8]{inputenc}

\usepackage{graphicx}

\usepackage{fancyhdr}

%Kopf- und Fußzeileneinstellungen
\fancypagestyle{lnifirstpage}{
% Löscht alle Kopf- und Fußzeileneinstellungen
\fancyhf{}

%Kopfzeile
\fancyhead[RO]{\small Einreichung für: <Konferenz/Workshop>,\linebreak%
Geplant als Veröffentlichung innerhalb der Lecture Notes in Informatics (LNI)
}
%Für den Herausgeber:
%\fancyhead[RO]{\small <Vorname Nachname [et. al.]> (Hrsg.): <Buchtitel>,\linebreak%
%Lecture Notes in Informatics (LNI), Gesellschaft für Informatik, Bonn <Jahr> \hspace{5pt} \thepage \hspace{0.05cm}}

%Linie unter Kopfzeile
\renewcommand{\headrulewidth}{0.4pt}
}

% Put in the short title (Kurztitle) here
\fancypagestyle{lni}{
\fancyhf{}
%Zu lange Titel müssen von den HerausgeberInnen gekürzt werden, Vorschläge der AutorInnen dazu sind herzlich willkommen.
\fancyhead[RO]{\small Der Kurztitel \hspace{5pt}
\thepage \hspace{0.05cm}}
\fancyhead[LE]{\hspace{0.05cm}\small \thepage \hspace{5pt}
%Bis zu drei AutorInnen werden alle angeführt, darüber hinaus wird nur die erste Autorin bzw. der erste Autor angeführt und alle Weiteren mit et al.\ abgekürzt.
%Zu lange AutorInnenlisten müssen von den HerausgeberInnen gekürzt werden.
Vorname1 Nachname1 und Vorname2 Nachname2}
\renewcommand{\headrulewidth}{0.4pt}
}

%if lstlistings is used
%better approach: use the minted package - see https://en.wikibooks.org/wiki/LaTeX/Source_Code_Listings#The_minted_package
\usepackage{listings}

\iflnienglish
\usepackage[figurename=Fig., tablename=Tab., font=small]{caption}
\else
\usepackage[figurename=Abb., tablename=Tab., font=small]{caption}
\fi

% Listingname heißt nun List.
\renewcommand{\lstlistingname}{List.}

\usepackage[T1]{fontenc}


%for demonstration purposes only
\usepackage[math]{blindtext}

%tweak \url{...}
\usepackage{url}
\urlstyle{same}
%improve wrapping of URLs - hint by http://tex.stackexchange.com/a/10419/9075
\makeatletter
\g@addto@macro{\UrlBreaks}{\UrlOrds}
\makeatother

%diagonal lines in a table - http://tex.stackexchange.com/questions/17745/diagonal-lines-in-table-cell
%slashbox is not available in texlive (due to licensing) and also gives bad results. Thus, we use diagbox
%\usepackage{diagbox}

\usepackage{booktabs}

%required for pdfcomment later
\usepackage{xcolor}

%for easy quotations: \enquote{text}
%also required by biblatex
\usepackage{csquotes}
\usepackage[
  backend=biber,
  style=LNI_biblatex/LNI
]{biblatex}

% new packages BEFORE hyperref
% See also http://tex.stackexchange.com/questions/1863/which-packages-should-be-loaded-after-hyperref-instead-of-before

%enable hyperref without colors and without bookmarks
\usepackage[
%pdfauthor={},
%pdfsubject={},
%pdftitle={},
%pdfkeywords={},
bookmarks=false,
breaklinks=true,
colorlinks=true,
linkcolor=black,
citecolor=black,
urlcolor=black,
pdfpagelayout=SinglePage,
pdfstartview=Fit
]{hyperref}
%enables correct jumping to figures when referencing
\usepackage[all]{hypcap}

%enable nice comments
\usepackage{pdfcomment}
\newcommand{\commentontext}[2]{\colorbox{yellow!60}{#1}\pdfcomment[color={0.234 0.867 0.211},hoffset=-6pt,voffset=10pt,opacity=0.5]{#2}}
\newcommand{\commentatside}[1]{\pdfcomment[color={0.045 0.278 0.643},icon=Note]{#1}}

%compatibality with TODO package
\newcommand{\todo}[1]{\commentatside{#1}}

%enable \cref{...} and \Cref{...} instead of \ref: Type of reference included in the link
\iflnienglish
\usepackage[capitalise,nameinlink]{cleveref}
%Nice formats for \cref
\crefname{section}{Sect.}{Sect.}
\Crefname{section}{Section}{Sections}
\crefname{figure}{Fig.}{Fig.}
\Crefname{figure}{Figure}{Figures}
\else
\usepackage[ngerman,capitalise,nameinlink]{cleveref}
\fi

%introduce \powerset - hint by http://matheplanet.com/matheplanet/nuke/html/viewtopic.php?topic=136492&post_id=997377
\DeclareFontFamily{U}{MnSymbolC}{}
\DeclareSymbolFont{MnSyC}{U}{MnSymbolC}{m}{n}
\DeclareFontShape{U}{MnSymbolC}{m}{n}{
    <-6>  MnSymbolC5
   <6-7>  MnSymbolC6
   <7-8>  MnSymbolC7
   <8-9>  MnSymbolC8
   <9-10> MnSymbolC9
  <10-12> MnSymbolC10
  <12->   MnSymbolC12%
}{}
\DeclareMathSymbol{\powerset}{\mathord}{MnSyC}{180}

%improve float placement
%source: http://people.cs.uu.nl/piet/floats/node1.html
%see also: http://tex.stackexchange.com/a/35130/9075
\renewcommand{\textfraction}{0.05}
\renewcommand{\topfraction}{0.95}
\renewcommand{\bottomfraction}{0.95}
\renewcommand{\floatpagefraction}{0.35}
\setcounter{totalnumber}{5}

%enable margin kerning
%currently does not work with pdflatex
%\usepackage{microtype}

% correct bad hyphenation here
\hyphenation{net-works semi-conduc-tor}


%%% Adapt to your needs from here

%Beginn der Seitenzählung für diesen Beitrag
\setcounter{page}{1}

\author{
Vorname1 Nachname1\footnote{Einrichtung/Universität, Abteilung, Anschrift, Postleitzahl Ort, emailadresse@author1} \, %
Vorname2 Nachname2\footnote{Einrichtung/Universität, Abteilung, Anschrift, Postleitzahl Ort, emailadresse@author2} \, %
und weitere Autorinnen und Autoren in der gleichen Notation}

\title{Der Titel des Beitrags, der auch etwas länger sein und über zwei Zeilen reichen kann}


\bibliography{paper}

\begin{document}
\maketitle

%hint by http://tex.stackexchange.com/a/30229/9075 and http://tex.stackexchange.com/a/247652/9075
\thispagestyle{lnifirstpage}
\pagestyle{lni}

%Auf Anzahl der AutorInnen setzen, damit die weitere Nummerierung der Fußnoten passt
%Dieser Befehl \verb|\setcounter{footnote}{2}| legt in dem Fall fest, dass 2 Fußnotennummern bereits für die AutorInnen verbraucht wurden, damit die darauf folgenden Fußnoten mit der richtigen Nummerierung (ab 3) fortfahren. Dieser Wert muss an die jeweilige Zahl an AutorInnen bzw. bereits verbrauchte Fußnoten angepasst werden, sofern im weiteren Verlauf Fußnoten verwendet werden.
\setcounter{footnote}{2}

\begin{abstract}
Die \LaTeX-Klasse \texttt{lni} setzt die Layout-Vorgaben für Beiträge in LNI Konferenzbänden um.
Dieses Dokument beschreibt ihre Verwendung und ist ein Beispiel für die entsprechende Darstellung.
Der Abstract ist ein kurzer Überblick über die Arbeit der zwischen 70 und 150 Wörtern lang sein und das Wichtigste enthalten sollte.
Die Formatierung erfolgt automatisch innerhalb des abstract-Bereichs.
\end{abstract}

\begin{keywords}
LNI Guidelines, \LaTeX Vorlage. Die Formatierung erfolgt automatisch innerhalb des keywords-Bereichs.
\end{keywords}

\section{Verwendung}
Die GI gibt unter \url{http://www.gi-ev.de/LNI} Vorgaben für die Formatierung von Dokumenten in der LNI Reihe.
Für \LaTeX-Dokumente werden diese durch die Dokumentenklasse \texttt{lni} realisiert.

Dieses Dokument basiert auf der offiziellen Dokumentation, aber wurde entsprechend der aktuellen Vorlage und den eingebundenen Paketen aktualisiert.
Die aktuelle Version wird unter \url{https://github.com/latextemplates/LNI} gepflegt.
Es werden (nach Möglichkeit) die aktuellen Versionen von der GI eingebunden.

Diese Dokumentation simplifiziert und setzt grundlegendes LaTeX-Wissen voraus.
Es werden generische Platzhalter an die entsprechenden Stellen (wie Beispielsweise Author) gesetzt und nicht weiter an anderer Stelle dokumentiert.

Dieses Template ist wie folgt gegliedert:
\Cref{sec:demos} zeigt Demonstrationen der Verbesserung von GitHub-LNI gegenüber der originalen Vorlage.
\Cref{sec:lniconformance} zeigt die Einhaltung der Richtlinien durch einfachen Text.
\Cref{sec:monographien} enthält Hinweise für Monographien. Diese kommen zum Einsatz, wenn \emph{kein} Tagungsbandbeitrag verfasst wird, sondern eine Dissertation oder ähnliches.

\section{Demonstrationen}
\label{sec:demos}
The symbol for powerset is now correct: $\powerset$ and not a Weierstrass p ($\wp$).

Brackets work as designed:
<test>

See \href{https://www.ctan.org/pkg/microtype}{microtype} in action here:
\blindtext

\section{Demonstration der Einhaltung der Richtlinien}
\label{sec:lniconformance}

\subsection{Literaturverzeichnis}
Der letzte Abschnitt zeigt ein beispielhaftes Literaturverzeichnis für Bücher mit einem Autor \cite{Ez10} und zwei AutorInnen \cite{AB00}, einem Beitrag in Proceedings mit drei AutorInnen \cite{ABC01}, einem Beitrag in einem LNI Band mit mehr als drei AutorInnen \cite{Az09}, zwei Bücher mit den jeweils selben vier AutorInnen im selben Erscheinungsjahr \cite{Wa14} und \cite{Wa14b}, ein Journal \cite{Gl06}, eine Website \cite{GI14} bzw.\ anderweitige Literatur ohne konkrete AutorInnenschaft \cite{XX14}.

Formatierung und Abkürzungen werden für die Referenzen \texttt{book}, \texttt{inbook}, \texttt{proceedings}, \texttt{inproceedings}, \texttt{article}, \texttt{online} und \texttt{misc} automatisch vorgenommen.
Mögliche Felder für Referenzen können der Beispieldatei \texttt{paper.bib} entnommen werden.
Andere Referenzen bzw.\ Felder müssen allenfalls nachträglich angepasst werden.

\subsection{Abbildungen}
\Cref{fig:demo} zeigt eine Abbildung.

\begin{figure}
  \centering
  Hier sollte die Graphik mittels \texttt{includegraphics} eingebunden werden.

  %\includegraphics[width=.8\textwidth]{filename}
  \caption{Demographik}
  \label{fig:demo}
\end{figure}

\subsection{Tabellen}
\Cref{tab:demo} zeigt eine Tabelle.

\begin{table}
\centering
\begin{tabular}{lll}
\toprule
Überschriftsebenen & Beispiel & Schriftgröße und -art \\
\midrule
Titel (linksbündig) & Der Titel \ldots & 14 pt, Fett\\
Überschrift 1 & 1 Einleitung & 12 pt, Fett\\
Überschrift 2 & 2.1 Titel & 10 pt, Fett\\
\bottomrule
\end{tabular}
\caption{Die Überschriftsarten}
\label{tab:demo}
\end{table}

\subsection{Programmcode}
Die LNI-Formatvorlage verlangt die Einrückung von Listings vom linken Rand.
In der \texttt{lni}-Dokumentenklasse ist dies für die \texttt{verbatim}-Umgebung realisiert.

\begin{verbatim}
public class Hello { 
    public static void main (String[] args) { 
        System.out.println("Hello World!"); 
    } 
} 
\end{verbatim}

Alternativ kann auch die \texttt{lstlisting}-Umgebung verwendet werden.

\Cref{L1} zeigt uns ein Beispiel, das mit Hilfe der \texttt{lstlisting}-Umgebung realisiert ist.

\lstset{basicstyle=\ttfamily}
\begin{lstlisting} [captionpos=b, caption=Beschreibung, label=L1, xleftmargin=0.5cm]
public class Hello { 
    public static void main (String[] args) { 
        System.out.println("Hello World!"); 
    } 
}
\end{lstlisting}

\subsection{Formeln und Gleichungen}

Die korrekte Einrückung und Nummerierung für Formeln ist bei den Umgebungen \texttt{equation} und \texttt{eqnarray} gewährleistet.

\begin{equation}
  1=4-3
\end{equation}
und
\begin{eqnarray}
  2=7-5\\
  3=2-1
\end{eqnarray}

\section{Hinweise für Monographien}
\label{sec:monographien}
Für Proceedings und Seminarbände ist die Seitennummerierung, wenn nicht anders vorgegeben, von den HerausgeberInnen vorzunehmen, die einzelnen Beiträge sind von deren AutorInnen also ohne Seitennummerierung vorzulegen; Monographien und Dissertationen sind von der Autorin/ vom Autor zu nummerieren:
\begin{itemize}
\item Vorwort, Inhaltsverzeichnis und Beiträge beginnen grundsätzlich auf einer rechten Seite, ggf.\ ist also eine Leerseite einzufügen.
\item Die Seitennummer des ersten Beitrages errechnet sich aus der Seitenzahl der \enquote{Startseiten} + 1
\item Die \enquote{Startseiten} (Muster siehe Anlage in der HerausgeberInnen-Information) bestehen aus:
\begin{itemize}
\item 1.\ Seite (rechts): GI-Logo
\item 2.\ Seite (links): leer
\item 3.\ Seite (rechts): Titelblatt
\item 4.\ Seite (links): Bibliographische Angaben
\item 5.\ Seite ff (rechts): Vorwort
\item Anschließend an das Vorwort auf neuer Seite (links oder rechts): Angaben zur Tagungsleitung, zum Programmkomitee oder Organisationsteam, etc. Hier können auch Logos von SponsorInnen und VeranstaltungspartnerInnen abgedruckt werden.
\item Beginnend auf einer rechten (ungerade nummerierten) Seite: Inhaltsverzeichnis
\end{itemize}
\end{itemize}

%No \bibliography{paper} allowed here - biblatex expects it at the preamble
%Run `biber paper` to generate the bibliography
\printbibliography

\end{document}
